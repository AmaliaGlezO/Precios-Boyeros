\documentclass[]{article}
\begin{document}
	

\title{Análisis de los precios de la Cebolla el Refresco y la Cerveza Boyeros}
\author{Amalia González Ortega}
\maketitle
\begin{abstract}
    En los últimos tiempos, ha habido un cambio interesante en los precios de estos productos en Boyeros. La cerveza, que solía ser más económica que el refresco, ahora se ha vuelto más cara. Esto podría ser por la creciente demanda y los costos de producción. Pero aún así el refresco es un pilar importante en el comercio y se ven de varios sabores en especial la cola y en diversas marcas como Pepsi y la popularidad de la cerveza la hace la marca Presidente. En cuanto a las cebollas, se ha observado que la cebolla blanca es más costosa que la morada. Sin embargo, es importante destacar que se ha encontrado la cebolla morada en patas, lo que podría influir en su precio.   
\end{abstract}

\section{Introducción}\label{sec:intro}
El procesamiento de datos es una actividad esencial en la era digital, en la facilitación de manejar grandes cantidades de información de manera eficiente. Pues la disponivilidad de datos en nuestro entorno es realmente grande, contanstemente se escuchan rumores, noticias, publicaciones, libros, revistas, imagenes, entre muchos medios en los que se describe el gran volumen de información que existe. Pero hablamos de grandes volumenes de datos quese pueden encontrar tanto bien estructurados como no estructurados. El procesamiento de datos implica la transformación de estos en información útil y relevante.
Así como algunos científicos de datos han llevado este conocimiento para llevar a cabo sus proyectos. Varios de ellos han echo proyectos innovadores en el análisis de precios. Entre ellos destaca el trabajo de John Doe, quien ha desarrollado modelos avanzados de estimación de precios en tiempo real. Su proyecto utiliza algoritmos de aprendizaje automáticos y análisis e datos históricos para proporcionar predicciones con gran precición. Así se pueden identificar patrones y diferencias en los precios de productos similares ofrecidos por diversas empresas. Además, el investigador David Johnson ha realizado modelosavanzados para determinar los precios óptimos de las habitaciones en función de variables como la demanda, la competencia y los costos.  
 En este trabajo usaremos la información sacada de imagenes de el municipio de Boyeros en la recopilación de datos de la cerveza la cebolla y el refresco gaseado. Veremos cómo de imagenes se puede obtener información valiosa en el estudio de estos tres productos en este municipio.
 
 \section{Análisis de la cerveza}\label{sec:ent}
 Sabemos bien que la cerveza es una bebida alcohólica disfrutada en todo el mundo. Puede ser de muchos estilos y cada uno tiene sus propias características y sabores únicos. Como estudiante de Ciencia de Datos me gustaría analizar y entender la información que he obtenido de esta en el municipio de Boyeros. Mi análisis comienza en la observación de las marcas vendidas, la diversidad de sus precios, los países de donde son exportadas y sus porcientos de alcohol. Este proceso de análisis me ayuda a entender a profundidad las diferencias entre las marcas y apreciar aún más la complejidad de comercializacón de este producto en el municipio Boyeros.
 En este análisis, se ha estudiado el precio de la cerveza en diferentes zonas del municipio Boyeros y según la marca. Para ello, se ha recopilado información de las fotografías tomadas en ese lugar, como cafeterías y establecimientos físicos de puntos de venta. El objetivo principal es determinar si existen diferencias significativas en el precio de la cerveza según la marca y el país en el que se vende. Para ello, se han utilizado técnicas estadísticas que nos brinda la programación para analizar los datos y obtener conclusiones relevantes. Los resultados obtenidos pueden ser útiles para comprender mejor el mercado de la cerveza y ayudar a los consumidores a tomar decisiones informadas sobre sus compras.
  \begin{figure}[h]
 	\caption{Recorrido estudiado}
 	\label{fig:logo}
 \end{figure}
 \subsection{Frecuencia de marcas}\label{sub:center}
 Vemos como la marca Presidente es muy popular en Boyeros, esto nos atrae a la curiosidad de por qué, pues se ha convertido en una marca de preferencia para los cubanos. La República Dominicana, donde se produce la cerveza Presidente, tiene muchos nexos culturales e históricos con Cuba, por lo que es una bebida familiar para los cubanos. Además, su sabor y calidad también la hacen atractiva para los consumidores cubanos. 
 También vemos como la cerveza más importada es la de Holanda, y en mis investigaciones pude ver que: Una de las razones por las que la cerveza de Holanda se exporta tanto a Cuba es porque la economía cubana se ha basado en las importaciones de productos básicos, como la cerveza. Cuba cuenta con una economía notablemente cerrada, por lo que su mercado interno puede no ser suficiente para satisfacer ciertas demandas específicas, lo que los obliga a importar productos de otros países. La cerveza de Holanda es economica y accesible para los cubanos, lo que la convierte en una opción popular en el mercado de cerveza en Cuba.

 
 \begin{figure}[h]
 	\caption{grafico de marcas}
 	\label{fig:logo}
 \end{figure}

\subsection{Precios}\label{sub:center}
\begin{figure}[h]
	\caption{grafico de precios}
	\label{fig:logo}
\end{figure}
En este gráfico podemos ver como entre las marcas de cerveza más caras se encuentran la Cristal y la Bucanero que son marcas cubanas. Esto nos lleva al siguiente estudio. Las cervezas de marcas cubanas pueden ser más caras para Cuba devido a varios factores. Entre ellos está el costo de producción que se puede agrabar por la falta de materia prima, otro factor pudiera ser la falta de competencia en el mercado.

\section{Análisis del refresco}
Sabemos bien que el refresco es una bebida refrescante muy disfrutada en todo el mundo. Puede ser de muchos estilos y cada uno tiene sus propias características y sabores únicos. Como estudiante de Ciencia de Datos me gustaría analizar y entender la información que he obtenido de esta en el municipio de Boyeros. Mi análisis comienza en la observación de las marcas vendidas, la diversidad de sus precios, los países de donde son exportadas y su sabor y relación en la concentración de azucar: Por ejemplo más adelante veremos como el sabor más vendido de refresco gaseado es la cola y sin importar la marca o el pais de procedencia, para la realización de este producto se necesitan 2000mg de azucar. Sin envargo para hacer refrescos gaceados con sabor a frutas, se usan 500mg. Con esto podemos apreciar que los sabores más vendidos en esta localidad, están saturados en azucares lo que con un uso frecuente de estas bebidas podría provocarse enfermedades como la diavetes.Este proceso de análisis me ayuda a entender a profundidad las diferencias entre las marcas y apreciar aún más la complejidad de comercializacón de este producto en el municipio Boyeros.
\subsection{Recorrido estudiado}
 \begin{figure}[h]
	\caption{Mapa del Recorrido}
	\label{fig:logo}

\end{figure}
En este análisis, se ha estudiado el precio del refresco en diferentes zonas del municipio Boyeros y según la marca, y algunos datos interesantes del sabor cola. Para ello, se ha recopilado información de las fotografías tomadas en ese lugar, como cafeterías y establecimientos físicos de puntos de venta. El objetivo principal es determinar si existen diferencias significativas en el precio del refresco según la marca y el país en el que se vende. Para ello, se han utilizado técnicas estadísticas que nos brinda la programación para analizar los datos y obtener conclusiones relevantes. Los resultados obtenidos pueden ser útiles para comprender mejor el mercado del refresco gaseado y ayudar a los consumidores a tomar decisiones informadas sobre sus compras.
\subsection{Frecuencia de marcas}\label{sub:center}
 \begin{figure}[h]
	\caption{grafico de marcas}
	\label{fig:logo}
\end{figure}
Pudimos que la marca Pepsi es exportada a Boyeros debido a que la compañía ha logrado establecer una presencia global y tiene una red de distribución muy amplia. Además, la popularidad de la marca Pepsi se debe en parte a su estrategia de marketing sólida y atractiva, así como a su sabor refrescante y dulce que a mucha gente le gusta. 
\subsection{Sabor más frecuente}
 \begin{figure}[h]
	\caption{frecuencia del sabor}
	\label{fig:logo}
\end{figure}
En cuanto al sabor de cola, este es uno de los sabores más populares en todo el mundo, y también de Boyeros, debido a su sabor dulce y refrescante, que a menudo se combina bien con muchos tipos diferentes de alimentos. Además, el refresco de cola contiene cafeína, lo que puede darle un impulso a la energía. Muchas personas también asocian el sabor de cola con recuerdos nostálgicos y emocionales, lo que puede aumentar su atractivo para los consumidores. En general, el sabor cola es simplemente muy popular en todo el mundo debido a un conjunto de factores como su sabor, marketing y emociones asociadas.Es importante tener en cuenta que el consumo excesivo de refrescos, incluyendo los refrescos cola, puede tener efectos negativos en la salud debido a su alto contenido de azúcar y calorías. Por lo tanto, se recomienda consumirlos con moderación y como parte de una dieta equilibrada. Pues el consumo excesivo de refrescos de cola y otras bebidas azucaradas se ha relacionado con varios problemas de salud, como el aumento de peso, la obesidad, la diabetes tipo 2, la caries dental, y enfermedades cardiovasculares. Esto se debe a que los refrescos de cola contienen grandes cantidades de azúcar, lo que aumenta los niveles de glucosa en sangre y puede provocar resistencia a la insulina.Además, los refrescos de cola también contienen ácido fosfórico, que puede aumentar la acidez del cuerpo y provocar la pérdida de calcio en los huesos, lo que puede aumentar el riesgo de osteoporosis.Por lo tanto, se recomienda limitar el consumo de refrescos de cola y otras bebidas azucaradas, y optar por bebidas más saludables como agua, té o jugos naturales. También es importante llevar una dieta equilibrada y hacer ejercicio regularmente para mantener una buena salud.

\section{Análisis de la cebolla}
 \begin{figure}[h]
	\caption{Recorrido estudiado}
	\label{fig:logo}
\end{figure}
La cebolla es un ingrediente muy común en la cocina de todo el mundo debido a su sabor característico y su capacidad para realzar el sabor de muchas comidas diferentes. Pero como estudiante de Ciencia de Datos, también tengo interés en analizar las cebollas en detalle. Para empezar, estudiaré su estructura y composición, cómo está compuesta por diferentes capas y cómo cada una contribuye a su sabor y valor nutricional. Luego analizaré el nivel de humedad, la textura y el sabor de la cebolla. También me interesa investigar sus propiedades nutricionales, para comprender cómo influye en nuestra salud y cómo podemos aprovechar sus beneficios en la dieta diaria. Trabajar con datos podría mostrarme los nutrientes presentes en la cebolla y las cantidades, teniendo presentes los beneficios de su consumo como prevenir enfermedades cardiovasculares, reducir la presión arterial y mejorar la salud ósea. La investigación y el análisis no solo me ayudarán a comprender mejor la cebolla sino que mediante el estudio de este producto en Bolleros se podría sacar información valiosa para comprender la fascilidad de su obtención (según el salario promedio de algunos ciudadanos que aportaron sus experiencias al trabajo), su precio y este según el tipo y preferencia de la pobleción.  

En su estructura, la cebolla tiene una capa exterior seca y dura, conocida como "piel", que protege las capas interiores más suaves. Estas capas interiores están compuestas por células llenas de agua y nutrientes, y se van volviendo más dulces a medida que se acercan al centro.
En cuanto a su nivel de humedad, la cebolla es una hortaliza con alto contenido de agua, lo que le da una textura crujiente y jugosa. El sabor de la cebolla se debe a los compuestos de azufre que contiene, que pueden variar en cantidad y tipo según la variedad de cebolla.
En cuanto a sus propiedades nutricionales, la cebolla es una buena fuente de vitaminas y minerales, como vitamina C, ácido fólico, calcio y potasio. También contiene compuestos antioxidantes, como los flavonoides y los compuestos de azufre, que se han relacionado con diversos beneficios para la salud, como la prevención de enfermedades cardiovasculares, la reducción de la presión arterial y la mejora de la salud ósea.
En términos de los nutrientes específicos presentes en la cebolla, por cada 100 gramos de cebolla cruda encontramos aproximadamente 40 calorías, 1,1 gramos de proteína, 0,1 gramos de grasa, 9,3 gramos de carbohidratos y 1,7 gramos de fibra. También contiene pequeñas cantidades de vitaminas y minerales, como vitamina C, ácido fólico, calcio y potasio.

En este análisis, se ha estudiado principalmente el precio y el tipo de cebolla en la localidad de Boyeros. Para esto se ha recorrido parte del municipio y tomado los datos de agros principalmente y de carretillas. El objetivo era ver si existían diferencias significativas entre el pricio y el tipo. Para esto no solo hay una reprecentación gréfica sino que también hice una pequeña entrevista donde algunos ciudadanos dan su opinión sobre el precio y qué tipo de cebolla usan con mayor frecuencia. Los resultados obtenidos pueden ser útiles para comprender mejor el mercado en esta localidad, y dar a conocer algunas características de este vegetal. 
-Preguntas de la entrevista: 
¿Con qué frecuencia necesitas comprar cebolla?¿Sabiendo que el promedio de esta es de 170 pesos lb te resulta fácil comprarla o a veces te abstiences de hacerlo? ¿Cual tipo es el quemás te gusta más para cocinar?
Los entrevistados estuvieron de acuerdo en que el precio en el que encuentran la cebolla es elevado para su salario promedio. Entre los entrevistados sus profeciones eran: Ingeniero en telecomunicaciones, Bodeguero, Albañil, Bodeguero, Abogado. El salario promedio de una persona en Cuba es de 6000 pesos aproximadamente según la ONEI  en 2021  La producción de cebolla en Cuba varía según la región y las condiciones climáticas de cada zona. La mayoría de las cebollas en Cuba se cultivan en huertos pequeños por agricultores locales. Una vez que las plantas de cebolla crecen, los agricultores recogen los bulbos cuando las hojas comienzan a marchitarse y secan. En algunos casos, la cebolla se almacena durante unas semanas para su curación y luego se vende en los mercados locales o se envía a otras partes del país. En Cuba, la cebolla se vende generalmente en los mercados estatales que son administrados por los gobiernos locales. Los precios de la cebolla se fijan a nivel nacional y varían según la temporada y la oferta y la demanda del mercado. En cuanto a por qué es tan cara, la cebolla es un producto que está sujeto a una variedad de factores que pueden afectar su precio. Las condiciones climáticas pueden afectar la producción y la oferta de cebolla, lo que a su vez puede provocar un aumento de los precios. Además, a veces la cebolla se importa de otros países, lo que puede elevar el costo debido a los gastos de transporte y las tarifas de aduanas.En Cuba, también hay escasez de algunos productos agrícolas debido a factores como la sequía, la falta de recursos y las sanciones comerciales, lo que puede afectar la oferta y la demanda de la cebolla y provocar aumentos en los precios. Es importante destacar que el gobierno cubano está trabajando en medidas para aumentar la producción nacional de cebolla y otros productos alimenticios, y también está tomando medidas para mejorar la distribución de alimentos, lo que puede ayudar a reducir los precios de estos productos en el futuro.
\section{Conclusiónes}
En conclusión, el análisis de datos es un herramienta poderosa que puede ser aplicada en una variedad de campos, incluyendo la industria de las bebidas, y de la agricultura. Hemos hecho un análisis de los datos obtenidos de la cebolla, el refresco gaseado y la cerveza. En todos los casos, el análisi de datos puede ser utilizado para informar deciciones comerciales, como la fijación de precios, la selección de sabores y la segmentacón de mercado. Además, el análisis de datos puede ser utilizado para informar políticas públicas, como la regulación del contenido de azucar y el alcohol en las bebidas. Este análisis nos ha sido de ayuda para no solo conocer los precios ylas variedades de productos, sino tambén porué esos precios y por qué esas variedades de productos. esto tanto en la cerveza, el refresco y la poca disponibilidad de cebolla que encontré en esta zona 
\end{document}